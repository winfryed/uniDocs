%! Author = 49176
%! Date = 16.08.2021

% Preamble
\documentclass[12pt]{article}

% Packages
\usepackage{amsmath}
\usepackage[utf8]{inputenc}
\usepackage[T1]{fontenc}
\usepackage[a4paper, left=2cm, right=2cm, top=2cm]{geometry}
\usepackage{amsfonts}
\usepackage{amssymb}

% Document
\begin{document}
    \title{Lineare Abbildung und Dimensionsformel}
    \maketitle
    \section{K-Lineare Abbildung}
    Eine Funktion  $f:V\rightarrow W$ \nolinebreak heißt K-Lineare Abbildung wenn gilt:

    \[\forall v,w\in V, \lambda \in K : f(v+w) = f(v) + f(w) \wedge f(\lambda v) = \lambda f(x)\]

    \section{Dimensionsformel}
    Definition: $dim(A)=dim(Kern(A))+dim(Bild(A))$

    \subsection{Rang und Defekt einer Matrix}

    Der Defekt von A ist : $Def(A) = dim(Kern(A))$ \newline
    \newline
    Der Rang von A ist : $rk(A)=dim(Bild(A))$ \newline
    \newline
    $\Rightarrow dim(A)=Def(A)+rk(a)$


    \subsection{Faustregeln (Ohne Anspruch auf Gültigkeit)}

    \begin{enumerate}
        \item $rk(A)\rightarrow$ "Anzahl an Linear unabhänigen Spalten einer Matrix"
        \item $Def(A) \rightarrow$ "Anzahl der Spalten - rk(A)"
        \item $dim(A) \rightarrow $ "Anzahl der Spalten"
    \end{enumerate}

    \newpage
    \section{Beispiele}

    \subsection{K-Lineare Abbildung}

    %Wir betrachten $U:=\{2\alpha+2\alpha X | \alpha \in \mathbb{R}\} \subseteq \mathbb{R}[X]$. \newline
    %Begründung warum U ein Untervektorraum des $\mathbb{R}$-Vektorraumes $\mathbb{R}[X]$ ist.

    Gegeben sei die Funktion $f:\mathbb{R} \rightarrow \mathbb{R}, f(x) = x^2$. \newline
    \newline
    Die Funktion ist nicht $\mathbb{R}$-Linear, denn \newline
    die Funktion ist $\mathbb{R}$-Linear wenn gilt:
    \[\forall v,w\in \mathbb{R}, \lambda \in \mathbb{R} : f(v+w) = f(v) + f(w) \wedge f(\lambda v) = \lambda f(v)\]
    Setzte $v,\lambda=2$ so gilt \newline
    \[f(\lambda v) = f(2\cdot2)=4^2=16\]
    \[\lambda f(v) = 2 f(2)=2\cdot 2^2=8\]
    \[\Rightarrow f(\lambda v) \neq \lambda f(v)\]
    \hfill $\Box$

    \subsection{Dimensionsformel}

    Sei $A:=\begin{pmatrix}1&1&1&1\\0&2&2&2\\0&3&3&3\\0&4&4&4\end{pmatrix}$ so folgt: \newline


    \begin{itemize}
        \item Dimension : $dim(A) = 4$
        \item Rang : $rk(A) = 3$
        \item Defekt: $Def(A) = 1$
    \end{itemize}





\end{document}