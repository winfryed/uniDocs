%! Author = $Matthias Weigt
%! Date = 17.08.2021

% Preamble
\documentclass[12pt]{article}

% Packages
\usepackage{amsmath}
\usepackage[utf8]{inputenc}
\usepackage[T1]{fontenc}
\usepackage[a4paper, left=2cm, right=2cm, top=2cm]{geometry}
\usepackage{wasysym}

% Document
\begin{document}
    \title{Komplexe Zahlen}
    \maketitle
    \section{Rechenregeln zu den komplexen Zahlen}

    Seien $z_1$ = 2 - 3i und $z_2$ = 4 + 5i komplexe Zahlen, so gilt: \newline

    \subsection{Addition / Subtraktion}

    $z_1 \pm z_2 = (a_1 + b_1 * i) \pm (a_2 + b_2 * i) = (a_1 \pm a_2) + (b_1 \pm b_2) * i$ \newline
    \newline
    Beispiel:
    \begin{itemize}
        \item $z_1 + z_2 = (2 + 4) + (-3 + 5) * i = 6 + 2i$
        \item $z_1 - z_2 = (2 - 4) + (-3 - 5) * i = -2 - 8i$
    \end{itemize}


    \subsection{Multiplikation}

    $ z_1 * z_2 = (a_1 * a_2 - b_1 * b_2) + (a_1*b_2 + a_2 *b_1) * i$ \newline
    \newline
    Beispiel:

    \begin{itemize}
        \item $z_1 * z_2 = (2 * 4 - -3 * 5) + (2 * 5 + 4 * (-3)) * i = 23 - 2i$
    \end{itemize}

    \subsection{Betrag}

    $|z_1| = \sqrt{a^2 + b^2}$  \newline
    \newline
    Beispiel:
    \begin{itemize}
        \item $|z_1| = \sqrt{2^2 + (-3)^2} = \sqrt{13}$ ; $\overline{z_1} = 2 + 3i$
    \end{itemize}

    \subsection{Kunjugat}

    $\overline{z_1} = a_1 - b_1 * i$ \newline
    \newline
    Beispiel:
    \begin{itemize}
        \item $|z_1| = \overline{z_2} = 4 - 5i$
    \end{itemize}

    \subsection{Weiteres}
    \begin{itemize}
        \item $ \overline{z_1 + z_2} = \overline{z_1} + \overline{z_2}$
        \item $ \overline{z_1 * z_2} = \overline{z_1} * \overline{z_2}$
        \item $|z_1 * z_2| = |z_1| * |z_2|$
    \end{itemize}


\end{document}