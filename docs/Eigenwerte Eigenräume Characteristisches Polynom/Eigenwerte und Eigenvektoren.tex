
\section{Eigenwerte und Eigenvektoren}

\subsection{Definiton Eigenwert, Eigenvektor}
Seien V ein K-Vektorraum und $f:V\rightarrowV$ ein Endomorphismus.
\begin{enumerate}[label=(\alph*)]
    \item Ein Skalar $\lambda \in K$ heißt Eigenwert von f, falls folgene Eigenschaft gilt: \[\exists x \in V \  \{ 0_V\} : f(x)=\lambda x \]
    \item Ist $\lambda \in K$ ein Eigenwert von f, so heißt jedes $x \in V \ \{0_V\}$ mit $f(x) = \lambda x$
    Eigenvektor zum eigenwert $\lambda $ von f und die Menge
    \[Eig_f(\lambda) :) \{x\in V | f(x) = \lambda x\}\] heißt Eigenraum zum Eigenwert von $\lambda$ von f.
\end{enumerate}