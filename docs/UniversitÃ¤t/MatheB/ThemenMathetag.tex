%! Author = $Matthias Weigt
%! Date = 18.08.2021

% Preamble
\documentclass[11pt]{article}

% Packages
\usepackage{amsmath}
\usepackage[utf8]{inputenc}
\usepackage[T1]{fontenc}
\usepackage[a4paper, left=2cm, right=2cm, top=2cm]{geometry}
\usepackage{wasysym}
\usepackage{amsfonts}
\usepackage{enumitem}
\usepackage{scalerel,amssymb}
\def\mcirc{\mathbin{\scalerel*{\circ}{j}}}
\def\msquare{\mathord{\scalerel*{\Box}{gX}}}

% Document
\begin{document}

    \title{Themen}
    \maketitle
    \section{$\msquare$ \hspace*{5mm} Funktionen ableiten $\longrightarrow$ Matthias}
    \subsection{Aufgabe}
    Ableiten : $f(x)=ln(x^2+3)sin(x)$
    \subsection{Aufgabe}
    Ableiten : $f(x)=(exp(x^2+1))^{\frac{1}{2}}$
    \subsection{Aufgabe}
    Ableiten : $f(x)=\frac{cos(x)}{ln(x)}$
    \subsection{Aufgabe}
    Ableiten : $f(x)=cos(ln(x^2+1))$
    \subsection{Aufgabe}
    Ableiten : $f(x)=\frac{exp(x^2+1)}{x^3}$
    \subsection{Aufgabe}
    Ableiten : $f(x)=\frac{cos(x)sin(x)}{ln(x)+3}+x^3$
    \subsection{Aufgabe}
    Ableiten : $f(x)=\sqrt{sin(x^2+3-sin(2x))}$

    \newpage\section{$\msquare$ \hspace*{5mm} Untervektorräume $\longrightarrow$ Collin}
    \section{$\msquare$ \hspace*{5mm} Stetigkeit $\longrightarrow$ Collin}
    \section{$\msquare$ \hspace*{5mm} Grenzwerte bestimmen $\longrightarrow$ Matthias}
    \subsection{Aufgabe}
    Grenzwert bestimmen : $\lim \limits_{n \to \infty}\frac{5n^2-\pi n-12}{31,3-n^2}$
    \subsection{Aufgabe}
    Grenzwert bestimmen : $\lim \limits_{n \to \infty}(sin(\frac{\pi}{n})+3\sqrt[n]{n}-1)$
    \subsection{Aufgabe}
    Grenzwert bestimmen : $\lim \limits_{n \to \infty}\sum\limits_{i=0}^{n}\frac{1}{(-3)^k}$
    \subsection{Aufgabe}
    Grenzwert bestimmen : $\lim \limits_{n \to \infty}\sum\limits_{i=0}^{n}\frac{1}{(-2)^k}$
    \subsection{Aufgabe}
    Grenzwert bestimmen : $\lim \limits_{n \to \infty}\sum\limits_{i=0}^{n}\frac{2}{2^k}$

    \section{$\msquare$ \hspace*{5mm} Eigenräume bestimmen $\longrightarrow$ Mark}
    \subsection{Aufgabe}
    Bestimme den Eigenraum zu jedem Eigenwert von  $A:=\begin{pmatrix}
                                                             3&1\\0&2
    \end{pmatrix}$

    \section{$\msquare$ \hspace*{5mm} Laplace Entwicklungssatz $\longrightarrow$ Mark}
    \section{$\msquare$ \hspace*{5mm} Matrix invertieren $\longrightarrow$ Collin}


\end{document}