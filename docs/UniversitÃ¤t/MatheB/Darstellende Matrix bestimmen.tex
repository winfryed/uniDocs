%! Author = $Matthias Weigt
%! Date = 17.08.2021

% Preamble
\documentclass[12pt]{article}

% Packages
\usepackage{amsmath}
\usepackage[utf8]{inputenc}
\usepackage[T1]{fontenc}
\usepackage[a4paper, left=2cm, right=2cm, top=2cm]{geometry}
\usepackage{wasysym}
\usepackage{amsfonts}

% Document
\begin{document}
    \title{Bestimmung der darstellenden Matrix}
    \maketitle
    
    \section{Gegeben:}

    \begin{itemize}
        \item 2 R-Vektorräume W und V, Basis b von W, Basis a von V.
        \item Eine R-Lineare Abbildung $f: W\rightarrow$ V.
    \end{itemize}


    \section{Gesucht:}
    \begin{itemize}
        \item Die abbildene Matrix $M^{b}_a(f)$
    \end{itemize}
    \section{Vorgehensweise}
    Gegeben seien die lineare Abbildung:
    \[f:\mathbb{R}^3\rightarrow V, \begin{pmatrix}\alpha\\\beta\\\gamma\\\end{pmatrix} \rightarrow \alpha - \gamma + (2\beta + 3\gamma)X\]
    sowie die Startbasis \[b := (e3,e2,e1 +e2) \: von \:\mathbb{R}^3\]
    und die Zielbasis \[a := (X ,1) \:von\: V\]
    wobei $e_i$ den i-ten Einheitsvektor in $\mathbb{R}^3$ bezeichnet (mit $i\in\{1,2,3\}$).

    \subsection{Alle Einträge der Startbasis b komponentenweise in f einsetzen (Hierbei sollten mehrere Gleichungen entstehen)}

    \begin{enumerate}
        \item $f (e3) = -1 +3X=3X-1$
        \item $f (e2) = 0 +2X=2X+0$
        \item $f (e1 +e2) = 1 +2X=2X+1$
    \end{enumerate}

    \subsection{Bilde für jede entstandene Gleichung eine Linearkombination aus der Zielbasis a.}

    \begin{enumerate}
        \item $3X -1=3a_1-a_2\rightarrow\begin{pmatrix} 3\\-1\\\end{pmatrix}$
        \item $2X +0=2a_1+0a_2\rightarrow\begin{pmatrix} 2\\0\\\end{pmatrix}$
        \item $2X +1=2a_1+1a_2\rightarrow\begin{pmatrix} 2\\1\\\end{pmatrix}$
    \end{enumerate}


    \subsection{Übertrage die Koeffizenten der Gleichung in eine Matrix}

    \[M^{b}_a(f)=\begin{pmatrix} 3&2&2\\-1&0&1\\\end{pmatrix},\]

\end{document}