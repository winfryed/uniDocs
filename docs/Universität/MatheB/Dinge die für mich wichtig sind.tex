%! Author = $Matthias Weigt
%! Date = 18.08.2021

% Preamble
\documentclass[11pt]{article}

% Packages
\usepackage{amsmath}
\usepackage[utf8]{inputenc}
\usepackage[T1]{fontenc}
\usepackage[a3paper, left=2cm, right=2cm, top=2cm]{geometry}
\usepackage{wasysym}
\usepackage{amsfonts}
\usepackage{enumitem}

% Document
\begin{document}
    \title{Sammlung der für mich am wichtigsten erscheinenden Definitionen}
    \maketitle
    \subsection{Quotientenregel}
    $f(x)=\frac{u(x)}{v(x)}\:\longrightarrow\:f'(x)=\frac{u'(x)\cdot v(x)-u(x)\cdot v'(x)}{v^2(x)}$
    \subsection{Untervektorraum Kriterium}
    Sei V ein K-Vektorraum. \newline
    U ist genau dann ein Untervektorraum von V wenn gilt:
    \begin{enumerate}[label=(\roman*)]
        \item $0_V\in U\Leftrightarrow U \neq \emptyset$
        \item $\forall \lambda \in K, v \in U: \lambda v \in U$
        \item $\forall v,w \in U: w+v \in U$
    \end{enumerate}
    \subsection{Linearitätskriterium}
    Seien W, V zwei K-Vekorräume.
    Eine Abbildung $f:W\rightarrow V$ ist Linear wenn gilt:
    \begin{enumerate}[label=(\roman*)]
    \item $\forall \lambda \in K, v\in W : f(\lambda v)=\lambda f(v)$
    \item $\forall v,w\in W : f(v+w) = f(v)+f(w)$
    \end{enumerate}
    \subsection{Dimensionsformel}
    Für eine lineare Abbildung $f:W\rightarrow V$ zwischen zwei K-Vektorräumen W und V gilt \newline\newline
    $dim_kW = rk_f + dim_Kkerf$
    \subsection{2x2 Matrix invertieren}
    Sei M eine Matrix mit $M=\begin{pmatrix}
                                 a&b\\c&d
    \end{pmatrix}$ so ist die inverse $M^{-1}=\frac{1}{ad-bc}\begin{pmatrix}
                                                                 d&-b\\-c&a
    \end{pmatrix}$
    \subsection{Laplace Formel}
    $detA=\sum\limits_{j=1}^{n}(-1)^{i+j}\cdot detA_{ij}\cdot\alpha_{ij}$
    \subsection{Charakteristisches Polynom}
    Das Charakteristisches Polynom ist definiert als:
    \[\chi_A(X):=det(XI_n-A)\]
    \subsection{Rechenvorschrift bestimmung Eigenraum zum Eigenwert $\lambda$}
    Zum berechnen eines Eigenraums zum Eigenwert $\lambda$ verwendet man die Formel:
    \[Eig_A(\lambda) := \mathcal{L}(\lambda I_n - A, 0_{K^n})\]
    \subsection{Lineare Unabhänigkeit}
    Sei V ein K-Vektorraum und $a:=(a_1,\dots,a_n)$ ein Tupel von Vektoren auf V. \newline
    Dann ist a linear unabhängig wenn gilt:
    \[
        \forall \lambda_1,\dots,\lambda_n\in K: \sum\limits_{i = 1}^{n}\lambda_ja_j=0_V\Rightarrow\lambda_1=\dots=\lambda_n=0_K
\]
    \subsection{Adjunkte Matrix}
    Für die Adjunkte Matrix gilt die Formel: $AA^*=A^*A=(detA)I_n$
    \subsection{Kern von f}
    Seien K ein Körper und $f:V\rightarrow W"$ eine K-Lineare Abbildung zwischen den K-Vektorräumen $V,W$. \newline
    Der Kern von v ist:
    \[
        ker\:f:=f^{-1}(\{0_W\})=\{v\inV | f(v)=0_W\}
    \]



\end{document}