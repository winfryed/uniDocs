\subsection{Rechenbeispiel}
Sei $A:=\begin{pmatrix}
            2&1&1\\
            1&1&0\\
            2&0&1
\end{pmatrix}$, $B:=\begin{pmatrix}
                            2&2\\
                            2&2\\
                            2&1
\end{pmatrix}$ \newline
\begin{table}[h!]
    \begin{tabular}{c|l}
        & $ \begin{pmatrix}
                2&2\\
                2&2\\
                2&1
        \end{pmatrix} $ \\[7mm]
        \hline \\ [-3mm]
        $ \begin{pmatrix}
              2&1&1\\
              1&1&0\\
              2&0&1
        \end{pmatrix} $ &
        $ \begin{pmatrix}
              2\cdot 2 + 1\cdot 2 + 1\cdot 2&2\cdot 2 + 1\cdot 2 + 1\cdot 1\\
              1\cdot 2 + 1\cdot 2 + 0\cdot 2&1\cdot 2 + 1\cdot 2 + 0\cdot 1\\
              2\cdot 2 + 0\cdot 2 + 1\cdot 2&2\cdot 2 + 0\cdot 2 + 1\cdot 1
        \end{pmatrix} $
    \end{tabular}\label{tab:table}
    \newline
    \newline
    \newline
    Somit gilt: \newline \newline
    $
        \begin{pmatrix}
              2\cdot 2 + 1\cdot 2 + 1\cdot 2&2\cdot 2 + 1\cdot 2 + 1\cdot 1\\
              1\cdot 2 + 1\cdot 2 + 0\cdot 2&1\cdot 2 + 1\cdot 2 + 0\cdot 1\\
              2\cdot 2 + 0\cdot 2 + 1\cdot 2&2\cdot 2 + 0\cdot 2 + 1\cdot 1
        \end{pmatrix} = \begin{pmatrix}
                                    8&7\\
                                    4&4\\
                                    6&5
        \end{pmatrix}
    $ \newline
    \newline
    Die Komponenten werden von außen nach innen paarweise multipliziert und die Paare addiert.
\end{table}