%! Author = $Matthias Weigt
%! Date = 18.08.2021

% Preamble
\documentclass[11pt]{article}

% Packages
\usepackage{amsmath}
\usepackage[utf8]{inputenc}
\usepackage[T1]{fontenc}
\usepackage[a4paper, left=2cm, right=2cm, top=2cm]{geometry}
\usepackage{wasysym}
\usepackage{amsfonts}
\usepackage{enumitem}

% Document
\begin{document}

    \section{Begrifflichkeiten bei Folgen}
    \subsection{Beschränktheit}
    Eine Reelle Folge $(a_n)_{n\in \mathbb{N}}$ heißt beschränkt, falls gilt:
    \[
        \exists C \in \mathbb{R}_{>0}:\forall n \in \mathbb{N}:|a_n|\leq C.
    \]
    \subsection{Konvergenz}
    \begin{enumerate}[label=(\alph*)]
        \item Sei $a\in \mathbb{R}$. Wir sagen, dass $(a_n)_{n\in\mathbb{N}}$ gegen a konvergiert, falls folgene Eigenschaft gilt: \newline
        \hspace*{10mm}$\forall\varepsilon>0:\exists n_0\in \mathbb{N}:\forall n \geq n_0:|a-a_n|<\varepsilon$
        \item Wenn ein $a\in\mathbb{R}$ exisitert, so dass $(a_n)_{n\in\mathbb{N}}$ gegen a kovergiert, nennt man die Folge kovergent.
        \item "nicht konvergent"$\Leftrightarrow$"divergent"
        \item Konvergiert $(a_n)_{n\in\mathbb{N}}$ gegen a, so schreiben wir \newline
        \hspace*{10mm} $\lim \limits_{n \to \infty}a_n:=a$ und $a_n\xrightarrow[n \rightarrow \infty]\: a$
    \end{enumerate}
    \subsubsection{Nullfolge}
    Konvergiert $(a_n)_{n\in\mathbb{N}}$ gegen 0, so nennen wir $(a_n)_{n\in\mathbb{N}}$ eine Nullfolge.
    \subsubsection{Differenzierbarkeit}
    Seien $a,b\in\mathbb{R}$ mit a<b. Seien $f:[a,b]\rightarrow\mathbb{R}$ eine Funktion und $x\in[a,b]$. \newline
    Dann heißt f differenzierbar in x, falls $\y\in\mathbb{R}$ existiert mit \newline\newline
    \hspace*{10mm} $\lim \limits_{n \to \infty}\frac{f(x+\varepsilon_n)-f(x)}{\varepsilon_n}$ \newline\newline
    für jede Nullfolge mit $(\varepsilon_n)_{n\in\mathbb{N}}$ mit $\varepsilon_n\neq 0 \forall n\in\mathbb{N}$. In diesem Fall
    heißt die y auch die Ableitung von $f$ in x und wird mit $f'(x)$ bezeichnet.

    \subsection{Teilfolgen und Häufungspunkte}
    \subsubsection{Teilfolgen}
    Seien $(a_n)_{n\in\mathbb{N}}$ und $(b_n)_{n\in\mathbb{N}}$ Folgen in $\mathbb{R}$. Wir nennen $(b_n)_{n\in\mathbb{N}}$ eine Teilfolge von $(a_n)_{n\in\mathbb{N}}$, falls\newline\newline
    \hspace*{10mm} $(b_n)_{n\in\mathbb{N}}=(a_{s(n)})_{n\in\mathbb{N}}$ \newline\newline
    für eine streng monoton steigende Funktion $s:\mathbb{N}\rightarrow\mathbb{N}$ gilt.
    \subsubsection{Häufungspunkt}
    Sei $(a_n)_{n\in\mathbb{N}}$ eine Folge in $\mathbb{R}$ und sei $a\in\mathbb{R}$. Dann heißt a Häufungspunkt von $(a_n)_{n\in\mathbb{N}}$, falls eine
    Teilfolge von $(a_n)_{n\in\mathbb{N}}$ existiert die gegen $a$ konvergiert.
\end{document}