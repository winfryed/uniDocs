Definiton 3.1.1 

Eine Funktion  $$f:V\rightarrow W$$ heißt K-Lineare Abbildung wenn gilt:

$$\forall v,w\in V, \lambda \in K : f(v+w) = f(v) + f(w) \wedge f(\lambda v) = \lambda f(x)$$ 


$$$$


Dimensionssatz:

$$dim(A)=dim(Kern(A))+dim(Bild(A))$$

Defekt:


$$dim(Kern(A)) = Def(A)$$

Rang:

$$dim(Bild(A))=rk(A)$$

Somit gilt:

$$dim(A)=Def(A)+rk(a)$$

Als Faustregel lässt sich festhalten:

1. $$rk(A)\rightarrow$$ "Anzahl an Linear unabhänigen Spalten einer Matrix"
2. $$Def(A) \rightarrow$$ "Anzahl der Spalten - rk(A)"
3. $$dim(A) \rightarrow$$ "Anzahl der Spalten"

Beispiel:


$$A:=\begin{pmatrix}
1&1&1&1\\
0&2&2&2\\
0&3&3&3\\
0&4&4&4
\end{pmatrix},$$

Somit folgt

Dimension : 

$$dim(A) = 4$$

Rang : 

$$rk(A) = 3$$

Defekt: 

$$Def(A) = 1$$

