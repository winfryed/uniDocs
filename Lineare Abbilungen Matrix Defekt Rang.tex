%! Author = 49176
%! Date = 16.08.2021

% Preamble
\documentclass[12pt]{article}

% Packages
\usepackage{amsmath}
\usepackage[utf8]{inputenc}
\usepackage[T1]{fontenc}
\usepackage[a4paper, left=2cm, right=2cm, top=2cm]{geometry}

% Document
\begin{document}
    \section{Definiton 3.1.1 - K-Lineare Abbildung}
    Eine Funktion  $f:V\rightarrow W$ \nolinebreak heißt K-Lineare Abbildung wenn gilt:

    \[\forall v,w\in V, \lambda \in K : f(v+w) = f(v) + f(w) \wedge f(\lambda v) = \lambda f(x)\]

    \section{Dimensionsformel}
    Definition: $dim(A)=dim(Kern(A))+dim(Bild(A))$

    \subsection{Rang und Defekt einer Matrix}

    Der Defekt von A ist : $Def(A) = dim(Kern(A))$ \newline
    \newline
    Der Rang von A ist : $rk(A)=dim(Bild(A))$ \newline
    \newline
    $\Rightarrow dim(A)=Def(A)+rk(a)$


    \subsection{Faustregeln (Ohne Anspruch auf totale Gültigkeit)}

    \begin{enumerate}
        \item $rk(A)\rightarrow$ "Anzahl an Linear unabhänigen Spalten einer Matrix"
        \item $Def(A) \rightarrow$ "Anzahl der Spalten - rk(A)"
        \item $dim(A) \rightarrow$ "Anzahl der Spalten"
    \end{enumerate}


    \subsection{Beispiel}



    \[A:=\begin{pmatrix}
             1&1&1&1\\
             0&2&2&2\\
             0&3&3&3\\
             0&4&4&4
    \end{pmatrix},\]

    Somit folgt

    Dimension :

    \[dim(A) = 4\]

    Rang :

    \[rk(A) = 3\]

    Defekt:

    \[Def(A) = 1\]



\end{document}